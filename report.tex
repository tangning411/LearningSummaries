\documentclass[12pt]{article}
%%---------------------------------------------------------------------
% packages
% geometry
\usepackage{geometry}
% font
\usepackage{fontspec}
\defaultfontfeatures{Mapping=tex-text}  %%如果没有它,会有一些 tex 特殊字符无法正常使用,比如连字符。
\usepackage{xunicode,xltxtra}
\usepackage[BoldFont,SlantFont,CJKnumber,CJKchecksingle]{xeCJK}  % \CJKnumber{12345}: 一万二千三百四十五
\usepackage{CJKfntef}  %%实现对汉字加点、下划线等。
\usepackage{pifont}  % \ding{}
% math
\usepackage{amsmath,amsfonts,amssymb}
% color
\usepackage{color}
\usepackage{xcolor}
\definecolor{EYE}{RGB}{199,237,204}
\definecolor{FLY}{RGB}{128,0,128}
\definecolor{ZHY}{RGB}{139,0,255}
% graphics
\usepackage[americaninductors,europeanresistors]{circuitikz}
\usepackage{tikz}
\usetikzlibrary{positioning,arrows,shadows,shapes,calc,mindmap,trees,backgrounds}  % placements=positioning
\usepackage{graphicx}  % \includegraphics[]{}
\usepackage{subfigure}  %%图形或表格并排排列
% table
\usepackage{colortbl,dcolumn}  %% 彩色表格
\usepackage{multirow}
\usepackage{multicol}
\usepackage{booktabs}
% code
\usepackage{fancyvrb}
\usepackage{listings}
% title
\usepackage{titlesec}
% head/foot
\usepackage{fancyhdr}
% ref
\usepackage{hyperref}
% pagecolor
\usepackage[pagecolor={EYE}]{pagecolor}
% tightly-packed lists
\usepackage{mdwlist}
\usepackage{verbatim}%comment命令的注释包
\usepackage{styles/iplouccfg}
\usepackage{styles/zhfontcfg}
\usepackage{styles/iplouclistings}

%%---------------------------------------------------------------------
% settings
% geometry
\geometry{left=2cm,right=1cm,top=2cm,bottom=2cm}  %设置 上、左、下、右 页边距
\linespread{1.5} %行间距
% font
\setCJKmainfont{Adobe Kaiti Std}
%\setmainfont[BoldFont=Adobe Garamond Pro Bold]{Apple Garamond}  % 英文字体
%\setmainfont[BoldFont=Adobe Garamond Pro Bold,SmallCapsFont=Apple Garamond,SmallCapsFeatures={Scale=0.7}]{Apple Garamond}  %%苹果字体没有SmallCaps
\setCJKmonofont{Adobe Fangsong Std}
% graphics
\graphicspath{{figures/}}
\tikzset{
    % Define standard arrow tip
    >=stealth',
    % Define style for boxes
    punkt/.style={
           rectangle,
           rounded corners,
           draw=black, very thick,
           text width=6.5em,
           minimum height=2em,
           text centered},
    % Define arrow style
    pil/.style={
           ->,
           thick,
           shorten <=2pt,
           shorten >=2pt,},
    % Define style for FlyZhyBall
    FlyZhyBall/.style={
      circle,
      minimum size=6mm,
      inner sep=0.5pt,
      ball color=red!50!blue,
      text=white,},
    % Define style for FlyZhyRectangle
    FlyZhyRectangle/.style={
      rectangle,
      rounded corners,
      minimum size=6mm,
      ball color=red!50!blue,
      text=white,},
    % Define style for zhyfly
    zhyfly/.style={
      rectangle,
      rounded corners,
      minimum size=6mm,
      ball color=red!25!blue,
      text=white,},
    % Define style for new rectangle
    nrectangle/.style={
      rectangle,
      draw=#1!50,
      fill=#1!20,
      minimum size=5mm,
      inner sep=0.1pt,}
}
\ctikzset{
  bipoles/length=.8cm
}
% code
\lstnewenvironment{VHDLcode}[1][]{%
  \lstset{
    basicstyle=\footnotesize\ttfamily\color{black},%
    columns=flexible,%
    framexleftmargin=.7mm,frame=shadowbox,%
    rulesepcolor=\color{blue},%
%    frame=single,%
    backgroundcolor=\color{yellow!20},%
    xleftmargin=1.2\fboxsep,%
    xrightmargin=.7\fboxsep,%
    numbers=left,numberstyle=\tiny\color{blue},%
    numberblanklines=false,numbersep=7pt,%
    language=VHDL%
    }\lstset{#1}}{}
\lstnewenvironment{VHDLmiddle}[1][]{%
  \lstset{
    basicstyle=\scriptsize\ttfamily\color{black},%
    columns=flexible,%
    framexleftmargin=.7mm,frame=shadowbox,%
    rulesepcolor=\color{blue},%
%    frame=single,%
    backgroundcolor=\color{yellow!20},%
    xleftmargin=1.2\fboxsep,%
    xrightmargin=.7\fboxsep,%
    numbers=left,numberstyle=\tiny\color{blue},%
    numberblanklines=false,numbersep=7pt,%
    language=VHDL%
    }\lstset{#1}}{}
\lstnewenvironment{VHDLsmall}[1][]{%
  \lstset{
    basicstyle=\tiny\ttfamily\color{black},%
    columns=flexible,%
    framexleftmargin=.7mm,frame=shadowbox,%
    rulesepcolor=\color{blue},%
%    frame=single,%
    backgroundcolor=\color{yellow!20},%
    xleftmargin=1.2\fboxsep,%
    xrightmargin=.7\fboxsep,%
    numbers=left,numberstyle=\tiny\color{blue},%
    numberblanklines=false,numbersep=7pt,%
    language=VHDL%
    }\lstset{#1}}{}
% pdf
\hypersetup{pdfpagemode=FullScreen,%
            pdfauthor={Haiyong Zheng},%
            pdftitle={Title},%
            CJKbookmarks=true,%
            bookmarksnumbered=true,%
            bookmarksopen=false,%
            plainpages=false,%
            colorlinks=true,%
            citecolor=green,%
            filecolor=magenta,%
            linkcolor=cyan,%red(default)
            urlcolor=cyan}
% section
%http://tex.stackexchange.com/questions/34288/how-to-place-a-shaded-box-around-a-section-label-and-name
\newcommand\titlebar{%
\tikz[baseline,trim left=3.1cm,trim right=3cm] {
    \fill [cyan!25] (2.5cm,-1ex) rectangle (\textwidth+3.1cm,2.5ex);
    \node [
        fill=cyan!60!white,
        anchor= base east,
        rounded rectangle,
        minimum height=3.5ex] at (3cm,0) {
        \textbf{\thesection.}
    };
}%
}
\titleformat{\section}{\Large\bf\color{blue}}{\titlebar}{0.1cm}{}
% head/foot
\setlength{\headheight}{15pt}
\pagestyle{fancy}
\fancyhf{}
%\lhead{\color{black!50!green}2014年秋季学期}
\chead{\color{black!50!green}Learning report}
%\rhead{\color{black!50!green}通信电子电路}
\lfoot{\color{blue!50!green}TangNing}
\cfoot{\color{blue!50!green}\href{http://vision.ouc.edu.cn/~zhenghaiyong}{CVBIOUC}}
\rfoot{\color{blue!50!green}$\cdot$\ \thepage\ $\cdot$}
\renewcommand{\headrulewidth}{0.4pt}
\renewcommand{\footrulewidth}{0.4pt}

%%---------------------------------------------------------------------
\begin{document}
%%---------------------------------------------------------------------
%%---------------------------------------------------------------------
% \titlepage
\title{\vspace{-2em}学习报告\vspace{-0.7em}}
\author{汤宁}
\date{\vspace{-0.7em}2015年10月\vspace{-0.7em}}
%%---------------------------------------------------------------------
\maketitle\thispagestyle{fancy}
%%---------------------------------------------------------------------
\maketitle
\tableofcontents 

\section{Linux基础学习}
\subsection{对Linux的一点小理解}
\begin{flushleft}
\subsubsection{目录树结构}
Linux中的所有数据都是以文件形态呈现的,以根目录为主,呈现分支状的目录结构。
\end{flushleft}
\subsubsection{“挂载”——{} 文件系统与目录树的关系}
\begin{flushleft}
进入一个目录读取该分区,进入的目录称为“挂载点”。
\end{flushleft}
\subsection{Ubuntu安装分区理解}
\begin{flushleft}
\slash{}:根目录,在linux中所有文件和目录都是从根目录开始的。\\
\slash{}swap:交换空间,当内存不足时,把一部分磁盘空间虚拟成内存使用从而解决内存不足的问题。\\
\slash{}home:用户目录,存放普通用户数据。\\
\slash{}opt:第三方软件安装目录(应用程序软件包)。\\
\slash{}boot(单系统时不用安装):存放系统启动相关程序(开机与内核文件)。
\subsection{Linux基本终端指令学习}
\subsubsection{查看文件与目录——{} ls}
ls [options]\\
1)当前目录下所有文件和文件夹。\\
2)-a:当前目录下所有文件(包括可见文件和隐藏文件)。\\
3)-l:当前目录下所有可见文件详细属性。\\
4)-al:当前目录下所有文件的详细属性。
\subsubsection{Linux文件权限相关指令}
\paragragh{\underline{改变所属用户组{} ——chgrp}\\
chgrp [-R] 账号名称 文件或目录\\
-R:递归更改,更改目录下的所有文件和目录。\\
\underline{改变文件所有者{} ——chown}\\
1.chown [-R] 账号名称 文件或目录\\
-R:递归更改,更改目录下的所有文件和目录。\\
2.chown root:root 文件名称\\
将该文件的所有者和用户组改回为root。\\
\underline{改变权限{} ——chmod}\\
1.数字类型改变权限:\\
chmod [-R] xyz(权限属性,分别代表所有者,用户组和其他人的权限) 文件或目录\\
-R:递归更改,更改目录下的所有文件和目录。\\

\fbox{r=4,w=2,x=1}\\
2.符号类型改变权限:\\
\fbox{u:所有者 g:用户组 o:其他人 a:所有}\\
chmod a+x 文件名\\
chmod a-x 文件名\\
chmod u=rwx,go=rx 文件名 \ldots\\
\begin{tabular}{|l|}
\hline
基本权限的作用:\\
一.对于目录:\\
r:当对目录具有权限时,可以查询该目录下文件名\\
w:可以更改目录结构列表\\
1.新建新的文件与目录\\
2.删除已存在的文件和目录\\
3.将已存在的文件和目录重命名\\
4.转移该目录内文件和目录位置\\
x:用户能否进入该目录成为工作目录\\
二.对于文件:\\
r:读取文件实际内容\\
w:编辑或新增修改文件内容,但不包含删除文件\\
x:可以被系统执行\\
\emph{PS:对于存放在某用户文件夹下的文件和目录,该用户在此文件或目录下具有rwx的完整权限}\\
\hline
\end{tabular}}\\
修改默认权限——{} umask\\umask——查看默认权限(默认值需要减掉的权限)\\umask -S 查看具体默认权限\\
默认权限属性:\\
1.用户创建``文件'',默认没有可执行(x)权限,最大为666\\
-rw-rw-rw-\\
2.用户新建``目录'',由于x与是否可以进入此目录有关,默认所有权限开放为777\\
drwxrwxrwx\\

设置文件隐藏属性——chattr\\
chattr [+-=][options] 文件或目录名\\
i:能使文件不能被删除,改名,设置连接,写入或添加数据,只有root能设置此属性。\\
a:设置后,文件只能增加数据,不能删除也不能修改,只有root能设置此属性。\\

显示文件隐藏属性——lsattr
}

\subsubsection{切换目录——{} cd}
cd [相对路径或绝对路径]\\
\underline{``."和``.."分别代表此层目录和上层目录。}\\
1)cd{} $\sim${} :回到自己的主文件夹。\\
2)cd ..:回到目前所在目录的上层目录。\\
3)cd -:回到刚才的目录。\\
4)cd ../目标目录:相对路径的写法,可以由当前目录到达与其属于同一上层目录的目标目录。
\subsubsection{复制,删除与移动——{} cp,rm,mv}
\paragragh{1.cp [options] 源文件{} 目标文件\\
1)-r:递归复制,可用来复制目录。\\
2)-p:连同文件属性一齐复制\\
2.rm [options] 文件或目录\\
-r:递归删除,常用于目录删除。\\}
3.mv 源文件{} 目标文件\\
可以用来改名。}
\subsubsection{新建目录与删除空目录——{} mkdir,rmdir}
\paragragh{mkdir [options] 目录名称\\
-p:可创建多层目录。\\rmdir [options] 空目录名称\\
-p:递归删除一系列空目录。\\}

\subsubsection{文件名的查找——{} find}
find / -name 文件名\\
find / -name xxx*   查找以xxx开头的文件\\
find / 目录 -name `*xxx*' 查找目录下文件名包含xxx的文件\\
find / -mtime+4 \\

\emph{PS:除了find命令外,还可以通过whereis和locate来查找,但whereis和find是从数据库中进行查找的,虽然查找速度较快但不一定能找到我们需要找的文件和目录,find虽然费时,但是从硬盘中直接查找。}\\

\begin{tabular}{|p{15cm}|}
\hline
mtime:内容更改时间\\
ctime:状态更改时间(权限和属性)\\
atime:文件内容被取用的时间\\
+n:n天前(不含n天本身)\\
-n:n天内(含n天本身)\\
n:n天之前的``一天之内”\\
\hline
\end{tabular}
\subsubsection{新建空文件——{} touch}
touch 文件名\\
新建了空文件,可以用`vim 文件名'打开该空文件并进入编辑模式添加数据。
\end{flushleft}
\subsection{vim程序编辑器}
\begin{flushleft}
vim 文件名\\
可进入一般模式,按`i'进入编辑模式,按`Esc'退回一般模式。\\
\underline{vim一般模式的基本按键功能:}\\
[Ctrl]+[f]——屏幕向下移动一页\\
[Ctrl]+[b]——屏幕向上移动一页\\
0——移动到这一行最前面的字符处\\
\${} ——移动到这一行最后面的字符处\\
G——移动到文件最后一行(nG移动到文件第n行)\\
gg——移动到文件第一行(相当于1G)\\
n+[Enter]——n为数字,光标向下移动n行\\
x,X——x为向后删除一个字符,X为向前删除一个字符\\
dd——删除光标所在那一行(ndd删除光标所在向下n行)\\
yy——复制光标所在那一行(nyy复制光标所在向下n行)\\
p,P——p为将已复制的数据在光标下一行粘贴,P为粘贴在光标上一行\\
u——复原前一个操作\\
[Ctrl]+[r]——重做上一个操作\\
:q——离开vim一般模式\\
:q!——强制离开不保存文件\\
:wq——保存后离开
\end{flushleft}
\subsection{学习Linux的心得体会}
在还没有接触Linux之前,我就已经听说过他的大名,但那个时候只是知道学习Linux有利于找工作罢了,而通过这段时间对Linux终端指令的学习,使我开始逐渐意识到Linux的强大之处。首先,Linux系统上的文件都是可以通过终端来进行更改设置的。之前我安装了Codeblocks和Opencv之后,对两者的全局编译链接进行了设置,设置后发现虽然每次关闭Codeblocks前都有保存过,但是下次打开后之前设置的都不见了,后来在老师的帮助下才发现Codeblocks安装包的所有者和用户组都是root,也就是说以我本人的名义是不能对Codeblocks进行设置更改的,于是我更改了安装包的所有者和用户组,问题才得以解决。通过这个小问题,使我对Linux的态度产生了极大的转变,我觉得Linux系统上文件都可以更改这一点就能很好的体现出它的开源性,虽然这样也会增大一不小心改动系统文件导致系统发生故障的风险。如果单单通过图形界面来操作Linux的话虽然有的时候也许会很方便,但是遇到比如权限一类的问题,图形界面是行不通的,很显然在这种情况下,终端指令无疑是最好的选择。此外,Linux的保护功能也很强大,我的电脑安装了Win7和Ubuntu双系统后,在Win7界面下只能看到它自己的磁盘分区而在Ubuntu界面下不仅可以看到它自己的还可以看到win7的磁盘分区,不同的是在Ubuntu下所有数据都是以文件形式存在的。Linux独特的目录树形式和挂载来读取分区也是比较有意思的地方。总之,这段时间对Linux的学习虽然还不是那么深入,但也算是收获颇丰,同时我还明白了一个道理:要想学好Linux还是得通过不断的实践才能更好的使用它,另外思考每个命令代表的意义也是很重要的。
\section{学习\LaTeX{}的心得体会}
\LaTeX{}是我最后安装的软件,我一直都觉得打印数学公式是一件很头疼的事情,最早的时候是从用word开始的,word中虽然带有大量数学公式可以编辑但是也有一部分不能修改,后来又知道有款专门打出数学公式的软件叫mathtype可以和word很好的结合,然而此款软件却需要付费。所以学习\Latex{} 之于我最主要的地方就是它能打出漂亮工整的数学公式,而如何使用\LaTeX{}来画出所需要的图表又是个大问题,在\LaTeX{}中比较简单的图表还是很容易可以得到的,而一旦遇到不规则表格,\LaTeX{}繁琐的代码就让我痛不欲生了。但是很显然它又有一个优势就是图形种类繁多,只有不会实现的没有实现不出来的。同时\LaTeX{}强大的排版方式也是word望尘莫及的,使用word对长篇文章进行排版时很多文本参数需要不断的手工调整,而相反使用\LaTeX{}排版时各种细节都可以统一规划设置,比较格式化。此外,\LaTeX{}与word相比显然排版更加正式,更适合论文和书籍的发表。总之,\LaTeX{}的功能还是很强大的,对于撰写学术论文无疑是最好的选择。










%%---------------------------------------------------------------------
\end{document}
